% framework provided by pgf-umlcd.sty, a set of macros for drawing UML class diagrams
% UML started on: 2023-04-05
% written by: Sebastian Schwindt

\documentclass{article}
\usepackage[margin=10mm,landscape,a2paper]{geometry}
\usepackage{hyperref}

\usepackage[
% school,
% simplified
]{pgf-umlcd}

\usepackage{listings}

\definecolor{listinggray}{gray}{0.92}
\lstset{ %
language=[LaTeX]TeX,
breaklines=true,
frame=single,
% frameround=tttt,
basicstyle=\footnotesize\ttfamily,
backgroundcolor=\color{listinggray},
keywordstyle=\color{blue}
}

\hypersetup{
  colorlinks=true,
  linkcolor=blue,
  anchorcolor=black,
  citecolor=olive,
  filecolor=magenta,
  menucolor=red,
  urlcolor=blue
}

% redefine UML package colors so that it hurts less in the eyes
\renewcommand{\umlfillcolor}{gray_light}


\begin{document}
\section*{\texttt{\detokenize{stochastic_calibration}}}

\begin{tikzpicture}
	\begin{package}{model_structure}
		\begin{class}[text width=10cm]{FullComplexityModel}{0,0}
			\attribute{model_dir: os.path }
			\attribute{control_file: str }
			\attribute{collocation_file: str }
			\attribute{res_dir: str }
			
			\operation{update_model_controls(new_parameter_values: dict, simulation_id: int): None}
			\operation{run_simulation(None): None}
		\end{class}
		\begin{class}[text width=10cm]{UserDefs}{0,10}
			\attribute{CALIB_PAR_SET: dict }
			\attribute{CALIB_PTS: numpy }
			\attribute{CALIB_TARGETS: list }
			\attribute{init_runs: int }
			\attribute{init_run_sampling: str}
			\attribute{IT_LIMIT: int}
			\attribute{MC_SAMPLES: int}
			\attribute{MC_SAMPLES_AL: int}
			\attribute{AL_SAMPLES: int}
			\attribute{AL_STRATEGY: str}
			\attribute{score_method: str}
			\attribute{SIM_DIR: str}
			\attribute{BME: None}
			\attribute{RE: None}
			\attribute{al_BME: None}
			\attribute{al_RE: None}
			
			\operation{assign_global_settings(all attributes): None}
			\operation{check_user_input(None): None}
			\operation{read_wb_range(read_range: str, sheet_name: str): pd.df}
		\end{class}
		\begin{object}[text width=10cm]{\detokenize{config_physics}}{0,13}
			\attribute{GRAVITY: 9.81 }
			\attribute{KINEMATIC_VISCOSITY: 10 ** -6}
			\attribute{WATER_DENSITY: 10. ** 3}
			\attribute{SED_DENSITY: 2650 }
		\end{object}
	\end{package}

	\begin{package}{telemac}
		\begin{class}[text width=10cm]{TelemacModel}{13,0}
			\inherit{FullComplexityModel}
			\attribute{calibration_parameters: dict}
			\attribute{control_file: str }
			\attribute{nproc: int }
			\attribute{slf_input_file: str}
			\attribute{tm_cas: str }
			\attribute{tm_results_file: str }
			\attribute{tm_xd: str }
			\attribute{tm_xd_dict: dict }
			\attribute{**gaia_cas: str }
			\attribute{**gaia_results_file: str }
						
			\operation{create_cas_string(param_name: str, value: var.): None}
			\operation{get_variable_value(slf_file_name: str, calibration_par: str, specific_nodes: np.array, save_name: str): np.array}
			\operation{rename_selafin(old_name=".slf", new_name=".slf"): None}
			\operation{rewrite_steering_file(param_name: str, updated_string: str, steering_module="telemac"): int}
			\operation{>> run_simulation}
			\operation{>> update_model_controls}
		\end{class}
		\begin{class}[text width=10cm]{TelemacUserDefs}{13,5}
			\inherit{UserDefs}
			\attribute{N_CPUS: int }
			\attribute{TM_CAS: str }
			\attribute{tm_xD: str }
			\attribute{GAIA_CAS: str }
			
			\operation{assign_calib_ranges(direct_par_df: pd.df, vector_par_df: pd.df, recalc_par_df: pd.df): None}
			\operation{check_user_input(None): None}
			\operation{>> assign_global_settings}
		\end{class}
		\begin{object}[text width=10cm]{\detokenize{config_telemac}}{13,11}
			\attribute{TM_TEMPLATE_DIR: os.path }
			\attribute{GAIA_PARAMETERS: pd.df}
			\attribute{TM2D_PARAMETERS: pd.df}
			\attribute{TM_TRANSLATOR: dict}
			\attribute{AL_RANGE = "A14:B22"}
			\attribute{MEASUREMENT_DATA_RANGE = "A23:B26"}
			\attribute{PRIOR_SCA_RANGE = "A32:B35"}
			\attribute{PRIOR_VEC_RANGE = "A38:B40"}
			\attribute{PRIOR_REC_RANGE = "A43:B44"}
			\attribute{ZONAL_PAR_RANGE = "A47:A49"}
			\attribute{RECALC_PARS: dict}
		\end{object}
	\end{package}
	
	\begin{object}[text width=10cm]{\detokenize{function_pool}}{19,16}
		\operation{append_new_line(file_name: str, text_to_append: str): None }
	\end{object}

	% associations for import directions
	\ association { \detokenize{function_pool} }{ TelemacUserDefs }{0..1}{}{0..*}{import}

\end{tikzpicture}

\end{document}
%\section{Basics}
%\subsection{Class with attributes and operations}
%Note: If you don't want to show empty parts in the diagrams, please
%use \texttt{simplified} option, e.g. \lstinline|\usepackage[simplified]{pgf-umlcd}|.\\
%\demo{class}
%
%\subsubsection{Visibility of attributes and operations}
%\demo[0.8]{visibility}
%
%\subsubsection{Abstract class and interface}
%\demo[0.5]{abstract-class}
%\demo[0.5]{interface}
%
%\subsubsection{Object}
%\demo[0.7]{object}
%Note: Object with rounded corners and methods are used in German school for didactic reasons. You get the rounded corners with \lstinline|\usepackage[school]{pgf-umlcd}|. If you need both in one document you can switch it with \lstinline|\switchUmlcdSchool| \\
%\switchUmlcdSchool
%\demo[0.7]{object}
%\demo[0.7]{object-include-methods}
%
%\subsubsection{Note}
%The \lstinline|\umlnote| use the same syntax as tikz command
%\lstinline|\node|, e.g. \lstinline|\umlnote[style] (name) at (coordinate) {text};|
%
%\demo[0.7]{note}
%
%\subsection{Inheritance and implement}
%\subsubsection{Inheritance}
%\demo{inheritance}
%\subsubsection{Multiple Inheritance}
%\demo{multiple-inheritance}
% 
%\subsubsection{Implement an interface}
%\demo{implement-interface}
%
%\subsection{Association, Aggregation and Composition}
%\subsubsection{Association} 
%\demo{association}
% 
%\subsubsection{Unidirectional association}
%\demo{unidirectional-association}
%
%\subsubsection{Aggregation}
%\demo{aggregation}
%
%\subsubsection{Composition}
%\demo{composition}
%
%\subsection{Package}
%\demo{package}
%
%\section{Customization}
%\subsection{Color settings}
%The color of digram is defined by \lstinline|\umltextcolor|, \lstinline|\umldrawcolor| and \lstinline|\umlfillcolor|, such as:
%
%\demo{color} 
%
%\section{Examples}
%\subsection{Abstract Factory}
%\example{abstract-factory}




